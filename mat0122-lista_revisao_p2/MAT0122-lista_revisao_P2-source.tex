\documentclass[11pt,reqno,a4paper]{amsart}

\usepackage{amsmath,amsthm}
\usepackage{setspace}
\usepackage[portuguese]{babel}
\usepackage[utf8]{inputenc}
\usepackage{datetime}
\usepackage{xfrac}
\usepackage{tabularx}
\usepackage{multicol}

\usepackage{fullpage}
\usepackage{setspace}

\usepackage{mathtools}

\makeatletter
\def\@setdate{\datename\ \@date}
%\def\@setthanks{\def\thanks##1{\par##1}\thankses}
\def\@setaddresses{\par
  \nobreak \begingroup
\footnotesize
  \def\author##1{\nobreak\addvspace\bigskipamount}%
  \def\\{\unskip, \ignorespaces}%
  \interlinepenalty\@M
  \def\address##1##2{\begingroup
    \par\addvspace\bigskipamount\indent
    \@ifnotempty{##1}{(\ignorespaces##1\unskip) }%
    {\scshape\ignorespaces##2}\par\endgroup}%
  \def\curraddr##1##2{\begingroup
    \@ifnotempty{##2}{\nobreak\indent{\itshape Current address}%
      \@ifnotempty{##1}{, \ignorespaces##1\unskip}\/:\space
      ##2\par}\endgroup}%
  \def\email##1##2{\begingroup
    \@ifnotempty{##2}{\nobreak\indent{\itshape Endereço eletrônico}%
      \@ifnotempty{##1}{, \ignorespaces##1\unskip}\/:\space
      \ttfamily##2\par}\endgroup}%
  \def\urladdr##1##2{\begingroup
    \@ifnotempty{##2}{\nobreak\indent{\itshape URL}%
      \@ifnotempty{##1}{, \ignorespaces##1\unskip}\/:\space
      \ttfamily##2\par}\endgroup}%
  \addresses
  \endgroup
}
\def\datename{\textit{Data}:}
\makeatother

\let\:=\colon
\let\epsilon=\varepsilon

\def\CC{{\mathbb C}}
\def\RR{{\mathbb R}}
\def\FF{{\mathbb F}}

\def\bfa{\mathbf{a}}
\def\bfb{\mathbf{b}}
\def\bfc{\mathbf{c}}
\def\bfd{\mathbf{d}}
\def\bfe{\mathbf{e}}
\def\bff{\mathbf{f}}
\def\bfi{\mathbf{i}}
\def\bfj{\mathbf{j}}
\def\bfp{\mathbf{p}}
\def\bfu{\mathbf{u}}
\def\bfv{\mathbf{v}}
\def\bfw{\mathbf{w}}
\def\bfx{\mathbf{x}}
\def\bfy{\mathbf{y}}
\def\bfX{\mathbf{X}}
\def\bfdelta{\boldsymbol\delta}
\def\bfzero{\mathbf{0}}
\def\dsone{\mathds{1}}

\def\({\left(}
\def\){\right)}  
\def\<{\langle}
\def\>{\rangle}

\def\Null{\mathop{\rm Null}}
\def\Ker{\mathop{\rm Ker}}
\def\Span{\mathop{\rm Span}}
\def\Aff{\mathop{\rm Aff}}
\def\GF{\mathop{\rm GF}}

\usepackage{enumitem}
\def\rmlabel{\upshape({\itshape \roman*\,})}
\def\RMlabel{\upshape(\Roman*)}
\def\alabel{\upshape({\itshape \alph*\,})}
\def\Alabel{\upshape({\itshape \Alph*\,})}
\def\nlabel{\upshape({\itshape \arabic*\,})} 
\newcommand\nlabelQ{\bf\upshape Q\arabic*}

\def\exemplo{\textbf{Exemplo.} }
\let\Exemplo\exemplo

\let\phi\varphi
\let\epsilon\varepsilon
\let\.\verb
\def\mystrut{\vrule height20pt depth12pt width0pt}

\newsymbol\nleq 2302

\def\resp#1{\noindent\textit{Resp.}: \framebox{\strut\hbox to #1{}}}
\def\respp#1#2{\noindent\textit{Resp.}: \framebox{\strut\hbox
      to #1{\hfill#2\hfill}}}
\def\Resp#1{\noindent\textit{Resposta}:\strut\\
  \framebox[.87\textwidth]{\phantom{\vrule height#1}}}
\def\Respcont#1{\noindent\textit{Resposta} (continuação):\strut\\
  \framebox[.87\textwidth]{\phantom{\vrule height#1}}}
\def\Respp#1#2{\noindent\textit{Resposta}:\strut\\
  \framebox[.87\textwidth]{#2\phantom{\vrule depth#1}}}
\def\Just#1{\noindent\textit{Justificativa}:\strut\\
\framebox[.87\textwidth]{\phantom{\vrule height#1}}}
\def\Assin#1{\noindent\textit{Assinatura}\strut:\\
\framebox[\textwidth]{\phantom{\vrule height#1}}}

\def\Draft#1{\noindent\textit{Rascunho}:\strut\\
  \framebox[.87\textwidth]{\phantom{\vrule height#1}}}

\def\sugestao#1{[\textit{Sugestão.} #1]}
\let\hint\sugestao
\def\observacao#1{[\textit{Observação.} #1]}
\let\NB\observacao

\begin{document}
\parindent=0pt

\title[MAC0122 Álgebra Linear I]%
{Álgebra Linear~I\\
  \bigskip\bigskip{\bf\textit{2$^{\text{\tiny o}}$ Semestre de 2022}}\\
  \bigskip\bigskip Exercícios de revisão para a P2
}  

\shortdate
\yyyymmdddate
\settimeformat{ampmtime}
\def\today{\number\year/\number\month/\number\day}
\date{\today, \currenttime}
\footskip=28pt

%\onehalfspace
\maketitle
\thispagestyle{empty} 
\pagestyle{plain}
%\doublespace
\onehalfspace

\noindent
Estes são uns exercícios de revisão para a segunda prova.  As
perguntas nos \textit{review questions} de \textbf{PNK} (Capítulos~6,
7, 8, 9 e~12) devem também ser revisados.

\medskip
\bigskip

\begin{enumerate}[label=\nlabelQ]\itemsep8pt

\item Prove o seguinte fato: não há $n+1$ vetores linearmente
  independentes em~$\FF^n$.  Procure dar uma prova simples e completa,
  que dependa o menos possível de fatos provados na disciplina (ou
  inclua a prova dos fatos que você usar, para sua prova ficar
  completa). 

\item Prove o seguinte fato: se~$M$ é uma matriz quadrada $n\times n$
  com posto~$n$, então~$M$ é inversível.  Procure dar uma prova
  simples e completa, que dependa o menos possível de fatos provados
  na disciplina (ou inclua a prova dos fatos que você usar, para sua
  prova ficar completa).

\item Seja $U\in\FF^{m\times n}$ uma matriz na forma escalonada,
  com~$m_1$ linhas não-nulas e~$m_2$ linhas nulas (naturalmente,
  $m=m_1+m_2$).  Quantas \textit{colunas} linearmente independentes
  tem a matriz~$U$?

\item Para cada uma das afirmações abaixo, diga se ela é verdadeira ou
  não.  Em cada caso, justifique sua resposta.
  \begin{enumerate}[label=\rmlabel]
  \item Seja $U\subset\FF^n$ um espaço vetorial.  Vale que
    $\FF^n=U\oplus U^\circ$.
  \item Seja $U\subset\FF^n$ um espaço vetorial.  Vale que
    $\dim U+\dim U^\circ=n$.
  \item Seja $U\subset\RR^n$ um espaço vetorial e
    $\<\,\cdot\,,\,\cdot\,\>$ um produto interno em~$\RR^n$.  Vale que
    $\RR^n=U\oplus U^\perp$.
  \item Seja $U\subset\RR^n$ um espaço vetorial e
    $\<\,\cdot\,,\,\cdot\,\>$ um produto interno em~$\RR^n$.  Vale que
    $\dim U+\dim U^\perp=n$.
  \end{enumerate}

\item Suponha que as matrizes $A\in\FF^{m\times n}$ e
  $B\in\FF^{n\times m}$ são tais $AB=I_m$, onde~$I_m\in\FF^{m\times
    m}$ é a matriz identidade. 
  \begin{enumerate}[label=\rmlabel]
  \item Prove que as colunas de~$B$ são linearmente independentes. 
  \item Prove que as colunas de~$A$ geram~$\FF^m$.
  \end{enumerate}

\item Seja $M$ uma matriz tal que $M^\top M$ é a matriz identidade.  É
  verdade que $MM^\top$ é a matriz identidade?  Por quê?  Há alguma
  hipótese simples sobre~$M$ que garanta a resposta positiva?
  
\item Fixe $b_1\in\GF(2)$ e~$b_2\in\GF(2)$ e considere `bitstrings'
  $\bfx=(x_1,\dots,x_6)\in\GF(2)^6$.  Tais bitstrings~$\bfx$ são
  chamados do tipo~$(b_1,b_2)$ se $x_1+x_3+x_5=b_1$ e
  $x_2+x_4+x_6=b_2$.  Quantos bitstrings do tipo~$(b_1,b_2)$ existem?
  
\item Considere a matriz
  \begin{equation}
    \label{eq:cycle}
    A=
    \begin{bmatrix}
      1&0&0&0&1\\
      1&1&0&0&0\\
      0&1&1&0&0\\
      0&0&1&1&0\\
      0&0&0&1&1\\
    \end{bmatrix}
  \end{equation}    
  \begin{enumerate}[label=\rmlabel]
  \item Prove que~$A$ é inversível como uma matriz real.
  \item Prove que~$A$ não é inversível como uma matriz
    sobre~$\GF(2)$.
  \end{enumerate}

\item Considere a
  matriz~$A_n=(a_{ij})_{1\leq i,j\leq n}\in\GF(2)^{n\times n}$ com
  \begin{equation}
    \label{eq:4}
    a_{ij}=
    \begin{cases}
      0 &\text{ se }i=j\\
      1 &\text{ caso contrário}.
    \end{cases}
  \end{equation}
  \begin{enumerate}[label=\rmlabel]
  \item Prove que~$A$ é inversível no caso em que~$n$ é par.
  \item Prove que~$A$ não é inversível no caso em que~$n$ é ímpar.
  \item Prove que~$A$ tem posto~$n-1$ no caso em que~$n$ é ímpar.
  \end{enumerate}

\item\label{Q:invert} Seja~$A\in\FF^{n\times n}$ uma matriz quadrada.
  Monte a matriz
  \begin{equation}
    \label{eq:AI}
    A'=
    \begin{bmatrix}
      A&I_n
    \end{bmatrix},
  \end{equation}
  onde~$I_n\in\FF^{n\times n}$ é a matriz identidade.  Suponha que
  executamos operações de escalonamento, e conseguimos
  transformar~$A'$ na matriz
  \begin{equation}
    \label{eq:IA}
    \begin{bmatrix}
      I_n&B
    \end{bmatrix}.
  \end{equation}
  Note que não apenas escalonamos~$A$, mas prosseguimos o processo
  até obter a identidade~$I_n$ (é fácil ver que isso é possível de
  se fazer caso obtenhamos no processo de escalonamento de~$A$ uma
  matriz~$U$ que tem todos seus elementos diagonais não nulos).
  Prove que~$B$ é a inversa de~$A$.  \NB{Esse fato sugere um
    algoritmo para se inverter~$A$.}

\item Suponha que executamos o algoritmo sugerido na
  Questão~\ref{Q:invert} para inverter uma matriz~$A$, mas a matriz
  escalonada que obtemos no meio do processo é tal que há zeros na
  diagonal.  Prove que~$A$ não é inversível.
    
\item Considere a equação $A\bfx=\bfb$ sobre~$\GF(2)$, onde
  \begin{equation}
    \label{eq:3}
    A=
    \begin{bmatrix}
      1&0&0&1\\
      1&1&1&1\\
      0&1&1&1\\
      1&1&1&0
    \end{bmatrix},\quad
    \bfx=
    \begin{bmatrix}
      x_1\\
      x_2\\
      x_3\\
      x_4
    \end{bmatrix}\text{\; e \;}
    \bfb=
    \begin{bmatrix}
      1\\
      0\\
      1\\
      1
    \end{bmatrix}.    
  \end{equation}
  Prove que esta equação não tem solução, considerando o vetor
  $\bfy=[1\;0\;1\; 1]^\top\in\GF(2)^4$.  \hint{Multiplique
    por~$\bfy^\top$.}
  
\item Considere a equação $U\bfx=\bfb$, onde~$U\in\FF^{m\times n}$ é
  uma matriz escalonada e~$\bfb=[b_1\;\dots\;b_m]^\top\in\FF^m$.
  Suponha que~$U$ tenha~$m_1$ linhas não-nulas e~$m_2$ linhas nulas
  (naturalmente $m=m_1+m_2$).  Prove que $U\bfx=\bfb$ admite uma
  solução se e só se $b_{m_1+1}=\dots=b_m=0$.

\item Considere a equação $A\bfx=\bfb$, onde $A\in\FF^{m\times n}$
  e~$\bfb\in\FF^m$.  Considere as duas afirmações abaixo:
  \begin{enumerate}
  \item[(A)] A equação $A\bfx=\bfb$ admite solução (isto é, existe
    $\bfx_0\in\FF^n$ tal que $A\bfx_0=\bfb$).
  \item[(B)] Existe $\bfy\in\FF^m$ tal que $\bfy^\top A=\bfzero$ e
    $\bfy^\top\bfb\neq0$. 
  \end{enumerate}
  Prove os seguintes dois fatos:
  \begin{enumerate}[label=\rmlabel]
  \item As afirmações~(A) e~(B) não podem valer simultaneamente.
  \item Necessariamente, ou a afirmação~(A) vale ou a afirmação~(B)
    vale.
  \end{enumerate}

\item Sejam $\bfu_1,\dots,\bfu_n\in\RR^n$ vetores dois a dois
  ortogonais e seja $\bfu=\sum_{1\leq i\leq n}\bfu_i$.  Prove que
  \begin{equation}
    \label{eq:2}
    \|\bfu\|^2=\sum_{1\leq i\leq n}\|\bfu_i\|^2.
  \end{equation}
  
\item Seja
    \begin{equation}
    \label{eq:H4}
    H={1\over2}
    \begin{bmatrix*}[r]
      1&1&1&1\\
      1&-1&1&-1\\
      1&1&-1&-1\\
      1&-1&-1&1\\
    \end{bmatrix*}, 
  \end{equation}
  e sejam $\bfb_1,\dots,\bfb_4\in\RR^4$ as colunas de~$H$.  Assim
  $H=[\bfb_1\mid\dots\mid\bfb_4]$.
  Seja~$V=\Span\{\bfb_2,\bfb_3,\bfb_4\}$ e
  $\bfe_1=[1\;0\; 0\; 0]^\top\in\RR^4$.  Encontre a projeção ortogonal
  \textit{sobre}~$V$ de~$\bfe_1$ e a projeção ortogonal \textit{a}~$V$
  de~$\bfe_1$.  Isto é, encontre $\bfe_1^{\parallel V}$
  e~$\bfe_1^{\perp V}$ de forma que
  $\bfe_1=\bfe_1^{\parallel V}+\bfe_1^{\perp V}$,
  $\bfe_1^{\parallel V}\in V$ e~$\<\bfe_1^{\perp V},\bfv\>=0$ para
  todo~$\bfv\in V$.
  
\item Sejam $\bfa_1,\dots,\bfa_m$ vetores em~$\RR^n$.  Suponha que
  $\bfx\in\RR^n$ é tal que $\<\bfx,\bfb\>=0$ para todo $\bfb\in\RR^n$
  tal que $\<\bfa_i,\bfb\>=0$ para todo $1\leq i\leq m$.  Prove que
  $\bfb\in\Span\{\bfa_1,\dots,\bfa_m\}$. 

\item Considere o sistema linear $A\bfx=\bfb$, onde
  $A\in\RR^{m\times n}$ e~$\bfb\in\RR^m$.  Queremos
  encontrar~$\hat\bfx\in\RR^n$ tal que
  \begin{equation}
    \label{eq:1}
    \|A\hat\bfx-\bfb\|=\min\{\|A\bfx-\bfb\|\:\bfx\in\RR^n\}.
  \end{equation}
  \begin{enumerate}[label=\rmlabel]
  \item Diga por que podemos supor que as colunas de~$A$ são
    linearmente independentes.  Nos itens a seguir, supomos que as
    colunas de~$A$ são linearmente independentes.
  \item Prove que~$A^\top A$ é inversível.
  \item Prove que~$\hat\bfx=(A^\top A)^{-1}A^\top\bfb$
    satisfaz~\eqref{eq:1}. 
  \end{enumerate}
  
\item Sejam
  \begin{equation}
    \label{eq:F}
    A=
    \begin{bmatrix}
      1&1\\
      1&0\\
    \end{bmatrix}\text{\; e \;}
    \bfu_0=
    \begin{bmatrix}
      0\\
      1\\
    \end{bmatrix}.
  \end{equation}
  Defina $\bfu_t$ ($t\geq1$) pondo $\bfu_t=A\bfu_{t-1}$.  Defina~$F_t$
  ($t\geq0$) pondo $F_t=t$ para~$t=0$ e~$t=1$ e $F_t=F_{t-1}+F_{t-2}$
  para~$t\geq2$.
  \begin{enumerate}[label=\rmlabel]
  \item Prove que $\bfu_t=[F_t\;\;F_{t-1}]^\top$ para todo~$t\geq1$.
  \item Deduza que, para todo~$t\geq0$, 
    \begin{equation}
      \label{eq:5}
      F_t={1\over\sqrt5}\(\phi_1^t-\phi_2^t\), 
    \end{equation}
    onde~$\phi_1=(1+\sqrt5)/2$ e~$\phi_2=(1-\sqrt5)/2$.
  \end{enumerate}

  
\end{enumerate}

\endgroup
\end{document}

%%% Local Variables:
%%% mode: latex
%%% eval: (auto-fill-mode t)
%%% eval: (LaTeX-math-mode t)
%%% eval: (flyspell-mode t)
%%% TeX-master: t
%%% End:
