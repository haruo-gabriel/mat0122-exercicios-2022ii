\usepackage{amsmath}
\usepackage{amssymb}
\newcommand{\Null}[1]{\mathrm{Null }{#1}}
\newcommand{\Row}[1]{\mathrm{Row }{#1}}
\newcommand{\Col}[1]{\mathrm{Col }{#1}}
\newcommand{\rank}[1]{\mathrm{rank }{#1}}
\newcommand{\dim}[1]{\mathrm{dim }{#1}}



\begin{document}

\section*{Insights para o exercício 68 item (ii)}
\subsection*{Intuição geométrica}
Reduza o exercício para o caso $U \in R^{2x3}$ e $U = b$.
Observe que temos 2 equações e 3 incógnitas, ou seja, temos 2 planos no $R^3$.

Agora, temos 2 casos: os planos se interceptam ou são paralelos.
Dois planos \textbf{nunca} se interceptarão em um ponto, ou seja, não há solução única.

Agora, abstraia essa ideia para uma dimensão a mais: para $R^{3x4}$.
Não há como 3 hiperplanos se interceptarem em apenas 1 ponto.
Logo, não há solução única.



{\subsection*{Caso geral}
}Seja $U \in R^{M \times N}$ uma matriz escalonada com $n$ linhas não-nulas.

Uma matriz escalonada implica que os vetores linha não-nulos formam uma base para o espaço das linhas de U.
Ou seja, $\{U_{1,*}, U_{2,*}, \dots, U_{n,*}\}$ é uma base para $\Row{U}$.

Pela definição de posto, $\rank{U} = \dim{\Col{U}}$.
Mas




# Im U = Col U => veja a definição de multiplicação matrix por vetor genérico e descubra o porque


\subsection*{Approach por teorema núcleo-imagem}
Pelo teorema núcleo-imagem,
\begin{align*}
    \dim \mathbb{F}^C = \dim \Null{U} + \dim \Col{U}\\
    \dim \mathbb{R}^N = \dim \Null{U} + \dim \Col{U}\\
    N = \dim \Null{U} + n\\
    \dim \Null{U} = N - n
\end{align*}

Se essa diferença for 0, então o espaço nulo de $U$ é vazio, logo a função associada a $U$ é injetora, logo existem $U$ e $b$ tal que o espaço solução de $U, b$ tem exatamente 1 elemento.

Em nosso caso, $N = 4$ e $n$, o posto do espaço das linhas, é no máximo 3.
Isso implica que $\Null{U} > 0$, impossível que $\Null{U}$ seja vazio.
Então, é impossível que exista apenas 1 solução.




\end{document}