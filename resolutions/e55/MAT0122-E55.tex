\documentclass[11pt,reqno,a4paper]{amsart}

\usepackage{setspace}
\usepackage[portuguese]{babel}
\usepackage[utf8]{inputenc}
\usepackage{dsfont}
\usepackage{amssymb}
\usepackage{amsmath}
\usepackage{xparse}

\usepackage{fullpage}
\usepackage{setspace}

\def\Assin#1{\noindent\textit{Assinatura}\strut\\%:\\
\framebox[\textwidth]{Gabriel Haruo Hanai Takeuchi\phantom{\vrule height#1}}}

\begin{document}
%newcommands
\newcommand{\Span}[1]{\mathrm{Span}(#1)}
\NewDocumentCommand{\comblin}{ O{\alpha} O{v} O{1} O{n}}{#1 _ #3  #2 _ #3 + \dots + #1 _ #4  #2 _ #4}

\parindent=0pt

\title{\textsl{MAT0122 Álgebra Linear I}\\\vspace{3\jot}
  Folha de solução}
\author[MAT0122 Folha de solução]{}

\maketitle
\thispagestyle{empty} 
\pagestyle{plain}
\onehalfspace

\textbf{Nome: Gabriel Haruo Hanai Takeuchi}\hfill
\textbf{Número USP: 13671636}\hspace{3cm}\null

\medskip
\Assin{1cm}

\medskip \textit{Sua assinatura atesta a autenticidade e
  originalidade de seu trabalho e que você se compromete a seguir o
  código de ética da USP em suas atividades acadêmicas, incluindo esta
  atividade.}

\medskip
\textbf{Exercício: E55}\hfill
\textbf{Data: 27/10/2022}\hspace{3cm}\null

\noindent\rule{\textwidth}{0.4pt}

\medskip
\noindent\textbf{SOLUÇÃO}

(i) Vamos provar que existe função linear requisitada. Precisamos que:

\textbullet\ Para qualquer $u \in \Span{S}$ e qualquer $\alpha \in \mathbb{F}$, $f(\alpha u) = \alpha f(u)$

Suponha um vetor $u \in \Span{S}$ tal que $u = \comblin[\beta]$. Logo, é necessário que
\begin{align*}
    f(\alpha u) &= f(\alpha(\comblin[\beta]))\\
    &= f(\comblin[\alpha \beta])\\
    &= f(\alpha \beta_1 v_1) + \dots + f(\alpha \beta_n v_n) &\text{(pela linearidade de f)}\\
    &= \alpha \beta_1 f(v_1) + \dots + \alpha \beta_n f( v_n) &\text{(pela linearidade de f)}\\
    &= \alpha \beta_1 \varphi (v_1) + \dots + \alpha \beta_n \varphi (v_n)
\end{align*}

\medskip

\textbullet\ Para quaisquer $u,v \in \Span{S}$, $f(u + v) = f(u) + f(v)$

Suponha então $u,w \in \Span{S}$ tal que $u = \comblin$ e $w = \comblin[\beta]$. Logo, é necessário que
\begin{align*}
    f(u+v) &= f(\comblin + \comblin[\beta])\\
    &= f((\alpha_1 + \beta_1) v_1 + \dots + (\alpha_n + \beta_n) v_n)\\
    &= f((\alpha_1 + \beta_1) v_1) + \dots + f((\alpha_n + \beta_n) v_n) &\text{(pela linearidade de f)}\\
    &= (\alpha_1 + \beta_1) f(v_1) + \dots + (\alpha_n + \beta_n) f(v_n) &\text{(pela linearidade de f)}\\
    &= (\alpha_1 + \beta_1) \varphi (v_1) + \dots + (\alpha_n + \beta_n) \varphi (v_n)
\end{align*}

Como mostrado acima, $f$ existe dessa forma.

\hrulefill

(ii) Vamos provar que $f$ é única. Portanto, suponha $g: \Span{S} \to W$ linear tal que $g(v_i) = \varPhi (v_i)$ para todo $1 \leq i \leq n$.

Observe que $g$ segue a mesma construção de linearidade que $f$ em (i).

\pagebreak

\textbullet\ Para qualquer $u \in \Span{S}$ e qualquer $\alpha \in \mathbb{F}$, $g(\alpha u) = \alpha g(u)$

Suponha um vetor $u \in \Span{S}$ tal que $u = \comblin[\beta]$. Logo, é necessário que
\begin{align*}
    g(\alpha u) &= g(\alpha(\comblin[\beta]))\\
    &= g(\comblin[\alpha \beta])\\
    &= g(\alpha \beta_1 v_1) + \dots + g(\alpha \beta_n v_n) &\text{(pela linearidade de g)}\\
    &= \alpha \beta_1 g(v_1) + \dots + \alpha \beta_n g( v_n) &\text{(pela linearidade de g)}\\
    &= \alpha \beta_1 \varphi (v_1) + \dots + \alpha \beta_n \varphi (v_n)\\
    &= \alpha \beta_1 f(v_1) + \dots + \alpha \beta_n f( v_n)\\
    &= f(\alpha u)
\end{align*}

\medskip

\textbullet\ Para quaisquer $u,v \in \Span{S}$, $g(u + v) = g(u) + g(v)$

Suponha então $u,w \in \Span{S}$ tal que $u = \comblin$ e $w = \comblin[\beta]$. Logo, é necessário que
\begin{align*}
    g(u+v) &= g(\comblin + \comblin[\beta])\\
    &= g((\alpha_1 + \beta_1) v_1 + \dots + (\alpha_n + \beta_n) v_n)\\
    &= g((\alpha_1 + \beta_1) v_1) + \dots + g((\alpha_n + \beta_n) v_n) &\text{(pela linearidade de g)}\\
    &= (\alpha_1 + \beta_1) g(v_1) + \dots + (\alpha_n + \beta_n) g(v_n) &\text{(pela linearidade de g)}\\
    &= (\alpha_1 + \beta_1) \varphi (v_1) + \dots + (\alpha_n + \beta_n) \varphi (v_n)\\
    &= (\alpha_1 + \beta_1) f(v_1) + \dots + (\alpha_n + \beta_n) f(v_n)\\
    &= f(u + v)
\end{align*}

Logo, $f = g$. Então, $f$ é uma função linear única.



\endgroup
\end{document}

%%% Local Variables:
%%% mode: latex
%%% eval: (auto-fill-mode t)
%%% eval: (LaTeX-math-mode t)
%%% eval: (flyspell-mode t)
%%% TeX-master: t
%%% End: