\documentclass[11pt,reqno,a4paper]{amsart}

\usepackage{setspace}
\usepackage[portuguese]{babel}
\usepackage[utf8]{inputenc}
\usepackage{dsfont}
\usepackage{amssymb}
\usepackage{amsmath}

\allowdisplaybreaks

\usepackage{fullpage}
\usepackage{setspace}

%newcommands
\newcommand{\Span}[1]{\mathrm{Span}(#1)}
\newcommand{\R}{\mathbb{R}}

\def\Assin#1{\noindent\textit{Assinatura}\strut\\%:\\
\framebox[\textwidth]{Gabriel Haruo Hanai Takeuchi\phantom{\vrule height#1}}}

\begin{document}
\parindent=0pt

\title{\textsl{MAT0122 Álgebra Linear I}\\\vspace{3\jot}
  Folha de solução}
\author[MAT0122 Folha de solução]{}

\maketitle
\thispagestyle{empty} 
\pagestyle{plain}
\onehalfspace

\textbf{Nome: Gabriel Haruo Hanai Takeuchi}\hfill
\textbf{Número USP: 13671636}\hspace{3cm}\null

\medskip
\Assin{1cm}

\medskip \textit{Sua assinatura atesta a autenticidade e
  originalidade de seu trabalho e que você se compromete a seguir o
  código de ética da USP em suas atividades acadêmicas, incluindo esta
  atividade.}

\medskip
\textbf{Exercício: 70}\hfill
\textbf{Data: 03/12/2022}\hspace{3cm}\null
\noindent\rule{\textwidth}{0.4pt}

\medskip
\noindent\textbf{SOLUÇÃO}

(i) Vamos continuar o processo de escalonamento da matriz.

A última matriz do exemplo 7.3.3 é a seguinte:
\begin{align*}
\begin{bmatrix}
	0 & 2 & 4 & 2 & 8\\
	2 & 1 & 0 & 5 & 4\\
	0 & 0 & 4 & -5 & -2\\
	0 & -2.5 & 0 & -10.5 & -2
\end{bmatrix}
\end{align*}

Vamos multiplicar a linha 0 por 1.25 e somá-la com a linha 3.
A matriz associada a essa transformação é a matriz $M_4$ tal que
\begin{align*}
	M_4
	\begin{bmatrix}
		0 & 2 & 4 & 2 & 8\\
		2 & 1 & 0 & 5 & 4\\
		0 & 0 & 4 & -5 & -2\\
		0 & -2.5 & 0 & -10.5 & -2
	\end{bmatrix}
	= 
	\begin{bmatrix}
		0 & 2 & 4 & 2 & 8\\
		2 & 1 & 0 & 5 & 4\\
		0 & 0 & 4 & -5 & -2\\
		0 & 0 & 5 & -8 & 8
	\end{bmatrix}
	\implies
	M_4 = 
	\begin{bmatrix}
		1 & 0 & 0 & 0\\
		0 & 1 & 0 & 0\\
		0 & 0 & 1 & 0\\
		1.25 & 0 & 0 & 1
	\end{bmatrix}
\end{align*}

Agora, vamos multiplicar a linha 2 da matriz transformada por -1.25 e somá-la com a linha 3.
A matriz associada a essa transformação é a matriz $M_5$ tal que
\begin{align*}
	M_5
	\begin{bmatrix}
		0 & 2 & 4 & 2 & 8\\
		2 & 1 & 0 & 5 & 4\\
		0 & 0 & 4 & -5 & -2\\
		0 & 0 & 5 & -8 & 8
	\end{bmatrix}
	= 
	\begin{bmatrix}
		0 & 2 & 4 & 2 & 8\\
		2 & 1 & 0 & 5 & 4\\
		0 & 0 & 4 & -5 & -2\\
		0 & 0 & 0 & -1.75 & 10.5
	\end{bmatrix}
	\implies
	M_5 = 
	\begin{bmatrix}
		1 & 0 & 0 & 0\\
		0 & 1 & 0 & 0\\
		0 & 0 & 1 & 0\\
		0  & 0 & -1.25 & 1
	\end{bmatrix}
\end{align*}

Conseguimos a matriz desejada.

(ii) A última matriz-transformação no exemplo 7.3.3 é a matriz $M'$ tal que
\begin{align*}
	M' =
	\begin{bmatrix}
		1 & 0 & 0 & 0\\
		0 & 1 & 0 & 0\\
		0.5 & -2 & 1 & 0\\
		0  & -2.5 & 0 & 1
	\end{bmatrix}
\end{align*}
Após isso, construímos as matrizes $M_4$ e $M_5$ que completam o escalonamento.
Perceba que ambas são matrizes inversíveis, logo a composição $M_5 \, M_4 \, M'$ também é inversível.
Isso implica que a multiplicação entre a composição acima e a matriz A do exemplo têm o mesmo espaço das colunas.

Portanto, $M = M_5 \, M_4 \, M'$ e $M A = U$. Segue a matriz $M$ abaixo (usei calculadora online de multiplicação de matrizes, o cálculo deve estar certo - acho).
\begin{align*}
	M = 
	\begin{bmatrix}
		1 & 0 & 0 & 0\\
		0 & 1 & 0 & 0\\
		0.5 & -2 & 1 & 0\\
		0.625 & 0 & -1.25 & 1
	\end{bmatrix}
\end{align*}

\hrulefill

(iii) A matriz $U'$ é a junção da matriz escalonada que obtivemos no item (ii) ao lado da matriz transformação $M$, também de (ii).
\begin{align*}
	U' = 
	\begin{bmatrix}
		0 & 2 & 4 & 2 & 8 & 1 & 0 & 0 & 0\\
		2 & 1 & 0 & 5 & 4 & 0 & 1 & 0 & 0\\
		0 & 0 & 4 & -5 & -2 & 0.5 & -2 & 1 & 0\\
		0 & 0 & 0 & -1.75 & 10.5 & 0.625 & 0 & -1.25 & 1
	\end{bmatrix}
\end{align*}

Ou seja, $B = M$.

\hrulefill

(iv) Não tenho a menor ideia de como provar formalmente, mas tentei no papel e deu certo.
Algumas observações que tive:

\medskip

Seja a matriz $M_1$ aquela primeira transformação de multiplicar a linha 1 por -2 e somar o resultado à linha 2. 
\begin{align*}
	M_1 A' = M_1 \begin{bmatrix}A & I\end{bmatrix} = \begin{bmatrix}M_1 A & M_1 I\end{bmatrix} = \begin{bmatrix}M_1 A & M_1\end{bmatrix}
\end{align*}

Seja $M_2$ a segunda transformação.
\begin{align*}
	M_2 \begin{bmatrix}M_1 A & M_1\end{bmatrix} = \begin{bmatrix}M_2 M_1 A & M_2 M_1 I\end{bmatrix}
\end{align*}

Eventualmente, temos
\begin{align*}
	\begin{bmatrix}M_k \ldots M_1 A & M_k \ldots M_1 I\end{bmatrix} = \begin{bmatrix}M A & M I\end{bmatrix} = \begin{bmatrix}U & M\end{bmatrix}
\end{align*}

Tudo isso se deve à essa tal propriedade de multiplicação de matrizes aninhadas que
\begin{align*}
	M \begin{bmatrix}N & O\end{bmatrix} = \begin{bmatrix}M N & M O\end{bmatrix}
\end{align*}


\endgroup
\end{document}

%%% Local Variables:
%%% mode: latex
%%% eval: (auto-fill-mode t)
%%% eval: (LaTeX-math-mode t)
%%% eval: (flyspell-mode t)
%%% TeX-master: t
%%% End: