
\documentclass[11pt,reqno,a4paper]{amsart}

\usepackage{setspace}
\usepackage[portuguese]{babel}
\usepackage[utf8]{inputenc}
\usepackage{dsfont}
\usepackage{amssymb}
\usepackage{amsmath}
\usepackage{systeme}
\usepackage{fullpage}
\usepackage{setspace}
\allowdisplaybreaks

%newcommands
\newcommand{\Span}[1]{\mathrm{Span}(#1)}
\newcommand{\R}{\mathbb{R}}

\def\Assin#1{\noindent\textit{Assinatura}\strut\\%:\\
\framebox[\textwidth]{Gabriel Haruo Hanai Takeuchi\phantom{\vrule height#1}}}

\begin{document}
\parindent=0pt

\title{\textsl{MAT0122 Álgebra Linear I}\\\vspace{3\jot}
  Folha de solução}
\author[MAT0122 Folha de solução]{}

\maketitle
\thispagestyle{empty} 
\pagestyle{plain}
\onehalfspace

\textbf{Nome: Gabriel Haruo Hanai Takeuchi}\hfill
\textbf{Número USP: 13671636}\hspace{3cm}\null

\medskip
\Assin{1cm}

\medskip \textit{Sua assinatura atesta a autenticidade e
  originalidade de seu trabalho e que você se compromete a seguir o
  código de ética da USP em suas atividades acadêmicas, incluindo esta
  atividade.}

\medskip
\textbf{Exercício: E78}\hfill
\textbf{Data: 10/12/2022}\hspace{3cm}\null

\noindent\rule{\textwidth}{0.4pt}

\medskip
\noindent\textbf{SOLUÇÃO}

Let's assume (because I have no idea how to prove it) the matrix $A \in \mathbb{R}^{m \times n}$ can be factored to be $A = Q R$, where $Q$ is an $m \times n$ column-orthogonal matrix and $R$ is an invertible matrix.

Let's initially consider the equation $A A^{intercal}$.
As we assumed early, $A = Q R$.

We are going to use the fact that if $A, B$ are matrices, then $(A B)^{\intercal} = B^{\intercal} A^{\intercal}$.

Therefore,
\begin{align*}
    A A^{\intercal} &= (Q R) (R^{\intercal} Q^{\intercal})\\
    &= Q I Q^{\intercal}\\
    &= Q Q^{\intercal} \\
    &= I &\text{[By the fact Q is orthonormal, as we've proven in exercise 77]}
\end{align*}


\endgroup
\end{document}

%%% Local Variables:
%%% mode: latex
%%% eval: (auto-fill-mode t)
%%% eval: (LaTeX-math-mode t)
%%% eval: (flyspell-mode t)
%%% TeX-master: t
%%% End: